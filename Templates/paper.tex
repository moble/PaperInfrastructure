\documentclass[reprint, aps, prd, letterpaper, noshowpacs, amsmath, %
amssymb, amsfonts, nofootinbib, floatfix, superscriptaddress, %
twoside]{revtex4-1}
\pdfoutput=1

%% This quiets a warning from bibtex
\bibliographystyle{apsrev4-1}

%auto-ignore

%% Basic font packages
\newcommand{\hmmax}{0}
\newcommand{\bmmax}{0}
\usepackage[T3,T1]{fontenc}
%\DeclareMathAlphabet{\mathpzc}{OT1}{pzc}{m}{it} % For \mathpzc fonts
\usepackage{mathrsfs} % For \mathscr fonts
\usepackage{bm} % For \bm fonts
\makeatletter
\@ifclassloaded{beamer}
  { % if this is a beamer document, use sans-serif fonts
    \typeout{UsePackages: Detected beamer}
    \usepackage{tgheros}
    \renewcommand*\familydefault{\sfdefault}
  }
  { % otherwise:
    \typeout{UsePackages: Did not detect beamer}
    \ifx\asybeamer\undefined % use times for articles
    \typeout{UsePackages: Detected article}
    \usepackage[varg]{txfonts} % For math
    \usepackage{tgtermes} % For text
    \else % use beamer fonts for asy documents with beamer
    \typeout{Fonts: Detected asy for beamer}
    \usepackage{tgheros}
    \renewcommand*\familydefault{\sfdefault}
    \fi
  }
\usepackage{microtype} % For nicer alignment of text
\def\MT@register@subst@font{
  \MT@exp@one@n\MT@in@clist\font@name\MT@font@list
  \ifMT@inlist@\else\xdef\MT@font@list{\MT@font@list\font@name,}\fi}
\makeatother

% %% Compact the text (smaller intra-word spacing)
% \AtBeginDocument{
%   \setlength{\spaceskip}
%   {0.875\fontdimen2\font
%     plus 0.925\fontdimen3\font
%     minus 1\fontdimen4\font}
% }

%% Spread out the text lines slightly
\linespread{1.0425}

%% No extra space after periods
%\frenchspacing

%% A few necessary packages
\usepackage{graphicx} %
\usepackage[x11names,svgnames,rgb]{xcolor} %
% \usepackage[all]{onlyamsmath} % Error on bad non-ams math
\usepackage{xspace}
%% And a few optional ones
\usepackage{braket}
\usepackage{accents}
%\usepackage{rotating} % 'sidewaystable' support
\usepackage{siunitx}
\usepackage{mathtools}
\DeclareSymbolFontAlphabet{\mathrm}{operators}

%% Define a few colors
\definecolor{CiteColor}{rgb}{0.18039, 0.18824, 0.57255}
%\definecolor{CiteColor}{rgb}{0.250, 0.250, 0.800}
\definecolor{UrlColor} {rgb}{0.741, 0.173, 0.000}
\definecolor{DarkUrlColor} {rgb}{0.500, 0.110, 0.000}
\definecolor{LinkColor}{rgb}{0.25098, 0.47843, 0.04706}
%\definecolor{LinkColor}{rgb}{0.000, 0.502, 0.118}


%% To get the current font parameters and linewidth in a LaTeX
%% document, put \ShowFont at the appropriate place.  The results will
%% be printed in the log file (and/or to screen).
\makeatletter %
\newcommand{\ShowFont}{%
  \typeout{The main font is \f@encoding \space \f@family \space %
    \f@series \space \f@shape \space at \f@size pt.}%
  \typeout{The math font sizes are \tf@size pt (main), \sf@size pt %
    (script), and \ssf@size pt (scriptscript).}%
  \typeout{The linewidth is \the\linewidth}} %
\makeatother %
%% See the LaTeX2e font selection guide for more details.
%% Note also that captions in revtex4 are typeset with \small; when
%%   most of the document is 10pt, \small is 9pt.
%% In beamer, the default font is cmss and the math font size 10.95pt.
%% In revtex4, we have the following widths:
%%            onecolumn   twocolumn
%%     10pt     510pt       246pt
%%     12pt     468pt       229pt
%% PRD editors resize things to fit in a 200x200pt box by default, but
%%   I like to keep figures bigger.


%% PGF/TIKZ
\usepackage{tikz} %
\usepackage{tikzscale} %
\usepackage{tikz-3dplot}
\usepackage{pgfplots} %
\usepackage{pgfplotstable} %
\pgfplotsset{compat=1.8} %
\usetikzlibrary{calc}
\usetikzlibrary{pgfplots.groupplots} %
%\usetikzlibrary{decorations.shapes} %
%\usetikzlibrary{decorations.markings} %
%% Automatically write figures as PDFs for import next time
\usetikzlibrary{pgfplots.external}
\tikzexternalize
% \tikzsetfigurename{Figure}
% \tikzset{external/force remake} % To force reprocessing of plots
%% Some nice defaults for me
\def\axisdefaultwidth{246pt} %
\def\axisdefaultheight{200pt} %
\pgfplotsset{scaled ticks=false} %
\pgfplotsset{axis line style={line width=1pt, black}} %
\pgfplotsset{tick style={line width=1pt, black}} %
\pgfplotsset{minor tick style={black}} %
\pgfplotsset{every axis/.append style={%
    axis on top, %
    legend cell align=left, %
    legend style={ %
      at={(0.03,0.97)}, %
      anchor=north west}, %
    no markers, %
  } %
} %
\pgfplotsset{every axis plot/.append style={thick, smooth}} %

%% Command to allow explicitly setting the file counter, as, e.g.,
%%   \tikzset{external/figure name={Figure}}
%%   \tikzset{external/set counter={Figure}{2}}
\makeatletter
\pgfqkeys{/tikz/external}{%
  set counter/.code 2 args={%
    \expandafter\gdef\csname c@tikzext@no@#1\endcsname{#2}%
  },%
}
\makeatother

% %% The following colors come from <http://colorbrewer2.org/>, which
% %% creates lists of nice colors for plots, etc.
% \definecolor{cycleA}{RGB}{228,26,28}
% \definecolor{cycleB}{RGB}{55,126,184}
% \definecolor{cycleC}{RGB}{77,175,74}
% \definecolor{cycleD}{RGB}{152,78,163}
% \definecolor{cycleE}{RGB}{255,127,0}
% \pgfplotscreateplotcyclelist{colorbrewer_set1}{%
%   {cycleA},{cycleB},{cycleC},{cycleD},{cycleE}%
% }

%% After I created the above chunk, a new pgf library was introduced,
%% which supplies colorbrewer cycles automatically.
\usepgfplotslibrary{colorbrewer}
%% Use with something like
% \begin{axis}[colorbrewer cycle list=Dark2]

%%% Local Variables:
%%% mode: latex
%%% TeX-master: "paper"
%%% End:

%auto-ignore

\makeatletter

%% Smaller space between section number and name
\def\@seccntformat#1{\csname the#1\endcsname.~}%

%% Smaller space between preceding text and new section headings
\def\section{%
  \@startsection {section}
  {1} {\z@} {0.55cm \@plus1ex \@minus .02ex}%
    {0.225cm} { \normalfont\bfseries \centering}%
}%
\def\subsection{%
  \@startsection {subsection}
  {2} {\z@ } {0.45cm \@plus 0.8ex \@minus 0.2ex}%
  {0.1125cm}{\normalfont \bfseries \centering }}
\def\subsubsection{%
  \@startsection {subsubsection}
  {3} {\z@ } {0.4cm \@plus 0.6ex \@minus 0.1ex}%
  {0.075cm}{\normalfont \it \centering }}

%% Headers and footers
\newcommand{\surnames}[1]{\def\@surnamelist{#1}\relax}
\surnames{\ }
\ifx \@shorttitle \@empty
  % \def\@oddhead{\small \MakeUppercase{\@title} \hfill}
  % Do nothing
\else
  \def\@oddhead{\small \MakeUppercase{\@shorttitle} \hfill}
\fi
\ifx \@surnamelist \@empty
  % Do nothing
\else
  \def\@evenhead{\small \MakeUppercase{\@surnamelist} \hfill}
\fi
\def\@oddfoot{\reset@font\hfil\thepage\hfil}%
\def\@evenfoot{\reset@font\hfil\thepage\hfil}%
\g@addto@macro\maketitle{\global\@specialpagetrue\gdef\@specialstyle{plain}}

%% Less hyphenation
\hyphenpenalty=1000  % Default: 50
%\tolerance=9000      % Default: 9999

%%%%%%%%%%%%%%%%%%%%%%%%%%%%%%%%%%%%%%%%%%%%%%%%%%%%%%%%%%%%%%%%%%%%%%
%% This is all for the hyperref package and nice colors

%% The following conflicts with \texorpdfstring in section headers
% %% First, patch a bug from revtex4 to allow use of \nameref
% \usepackage{xpatch}% http://ctan.org/pkg/xpatch
% \makeatletter
% \xpatchcmd{\@ssect@ltx}{\@xsect}%
%   {\edef\@currentlabelname{#8}\@xsect}{}{}% Patch \<section>*
% \xpatchcmd{\@sect@ltx}{\@xsect}%
%   {\edef\@currentlabelname{#8}\@xsect}{}{}% Patch \<section>
% \makeatother

\usepackage[colorlinks, plainpages=false,
hyperfigures=true]{hyperref}
% \usepackage[pagebackref=true, colorlinks, plainpages=false,
% hyperfigures=true]{hyperref}
% \renewcommand*{\backref}[1]{}
% \renewcommand*{\backrefalt}[4]{%
%   \ifcase #1 \or (Cited on p.~#2.) \else %
%   (Cited on pp.~#2.) \fi %
% }
\hypersetup{linkcolor=LinkColor}
\hypersetup{citecolor=CiteColor}
\hypersetup{urlcolor=UrlColor}
\hypersetup{setpagesize=false}
% \hypersetup{unicode} % Clashes with '\G' command defined in Macros
\hypersetup{pdfborder=0 0 0}
%%%%%%%%%%%%%%%%%%%%%%%%%%%%%%%%%%%%%%%%%%%%%%%%%%%%%%%%%%%%%%%%%%%%%%

\makeatother

%\numberwithin{equation}{section}

%%% Local Variables:
%%% mode: latex
%%% TeX-master: "paper"
%%% End:

%auto-ignore

%%%%%%%%%%%%%%%%
%% Math forms %%
%%%%%%%%%%%%%%%%
\let\Originalddefinition\d
\renewcommand{\d}{\ensuremath{\mathrm{d}}}
\let\Originaledefinition\e
\newcommand{\e}{\ensuremath{\mathrm{e}}}
\let\Originalidefinition\i
\renewcommand{\i}{\ensuremath{\mathrm{i}}}

\newcommand{\scri}{\ensuremath{\mathscr{I}}}
\newcommand{\scriplus}{\ensuremath{\mathscr{I}^{+}}}
\newcommand{\scriminus}{\ensuremath{\mathscr{I}^{-}}}
\newcommand{\Mirr}{\ensuremath{M_{\text{irr}}}}
\newcommand{\Eadm}{\ensuremath{M_{\text{ADM}}}}
\newcommand{\Jadm}{\ensuremath{J_{\text{ADM}}}}
\newcommand{\ADMMass}{\Eadm}
\newcommand{\IrrMass}{\Mirr}
\newcommand{\tr}{\ensuremath{t_\text{ret}}}
\newcommand{\rt}{\ensuremath{r_{\ast}}}
\DeclareMathOperator{\arccot}{arccot}
\DeclareMathOperator*{\argmin}{arg\,min}
\DeclareMathOperator*{\argmax}{arg\,max}

%% Units
\let\Originalcdefinition\c
\renewcommand{\c}{\mathrm{c}}
\newcommand{\G}{\mathrm{G}}
\newcommand{\MSun}{\ensuremath{M_\odot}\xspace}
\newcommand{\Sun}{\MSun}
\newcommand{\Mtot}{\ensuremath{M_\text{tot}}}
\DeclareSIUnit{\strain}{strain}
\DeclareSIPrePower{\root}{1/2}
\DeclareSIUnit{\parsec}{pc}
\DeclareSIUnit{\yr}{yr}
\DeclareSIUnit{\year}{yr}
\DeclareSIUnit{\lightyear}{ly}
\DeclareSIUnit{\SolarMass}{\ensuremath{\MSun}}
\DeclareSIUnit{\Mass}{\ensuremath{M}}
% \newcommand{\strain}{\text{strain}}
% \newcommand{\parsec}{\text{pc}}
% \newcommand{\SolarMass}{\MSun}

%% abs, norm, ceil, floor, avg, etc.
\newcommand{\abs} [1]{\left\lvert{#1}\right\rvert}
\newcommand{\norm}[1]{\left\lVert{#1}\right\rVert}
\newcommand{\ceil}[1]{\left\lceil{#1}\right\rceil}
\newcommand{\floor}[1]{\left\lfloor{#1}\right\rfloor}
\newcommand{\avg}[1]{\left\langle{#1}\right\rangle}
\newcommand{\co}[1]{\ensuremath{\bar{#1}}}
\DeclareMathOperator*{\RS}{RiemannSymmetrize}
\DeclareMathOperator{\sgn}{sgn}

%% Relational operators
\newcommand{\defined}{\coloneqq}
\newcommand{\identically}{\equiv}
\newcommand{\roughly}{\mathord{\sim}} % Different from \sim in spacing
%\newcommand{\corresponds}{\triangleq}
\newcommand{\asymptoticallyequal}{\simeq}

%% Wigner matrices, rotations, etc.
\newcommand{\D}{\ensuremath{\mathfrak{D}} }
\newcommand{\R}{\ensuremath{\mathcal{R}} }
% \newcommand{\sYlm}[1]{\ensuremath{\prescript{}{s}{Y}_{#1}}}
% \newcommand{\mTwoYlm}[1]{\ensuremath{\prescript{}{-2}{Y}_{#1}}}
\newcommand{\sYlm}[1]{\ensuremath{\scripts{_{s}}{Y}{_{#1}}}}
\newcommand{\sYbarlm}[1]{\ensuremath{\scripts{_{s}}{\bar{Y}}{_{#1}}}}
\newcommand{\mTwoYlm}[1]{\ensuremath{\scripts{_{-2}}{Y}{_{#1}}}}
\newcommand{\mTwoYbarlm}[1]{\ensuremath{\scripts{_{-2}}{\bar{Y}}{_{#1}}}}
%\newcommand{\Rotated}[1]{\ensuremath{\accentset{\frown}{#1}}}
%\newcommand{\Rotated}[1]{\ensuremath{\breve{#1}}}
%\newcommand{\Rotated}[1]{\ensuremath{#1'}}
\DeclareSymbolFont{tipa}{T3}{tipa}{m}{n}
\DeclareMathAccent{\ibreve}{\mathalpha}{tipa}{'020}
\newcommand{\Rotated}[1]{\ensuremath{\ibreve{#1}}}
\newcommand{\Rotation}[1]{\ensuremath{\mathbf{#1}}}
%\newcommand{\Generator}[1]{\ensuremath{\mathfrak{\MakeLowercase{#1}}}}
\newcommand{\Generator}[1]{\ensuremath{\MakeUppercase{#1}}}
\newcommand{\GeneratorVec}[1]{\ensuremath{\vec{\MakeLowercase{#1}}}}

\newcommand{\SO}[1]{\ensuremath{\mathit{SO}(#1)}}
\newcommand{\so}[1]{\ensuremath{\mathfrak{so}(#1)}}
\newcommand{\SU}[1]{\ensuremath{\mathit{SU}(#1)}}
\newcommand{\su}[1]{\ensuremath{\mathfrak{su}(#1)}}


%% Metric perturbation
\newcommand{\Background}[1]{\underaccent{\frown}{#1}}
\newcommand{\h}{\ensuremath{h} }
\newcommand{\hp}{\ensuremath{h_{+}} }
\newcommand{\hc}{\ensuremath{h_{\times}} }
\newcommand{\htilde}{\ensuremath{{\tilde{h}}}}
\newcommand{\Schw}{\ensuremath{\Background{R}} }
\newcommand{\GW}{\tilde{R}}
\newcommand{\Schwg}{\ensuremath{\Background{g}} }
\newcommand{\GWg}{\tilde{g}}
\newcommand{\SchwScalar}{\ensuremath{{\Background{\boxdot}}} }

%% Frequencies, discretization, DA stuff...
\newcommand{\fsamp}{\ensuremath{f_{\mathrm{s}}}}
\newcommand{\fNy}{\ensuremath{f_{\mathrm{Ny}}}}
\newcommand{\discretize}{\rightsquigarrow}
%%\newcommand{\MM}{\text{MM}} %% See below
\newcommand{\parameters}{\vec{\lambda}}
\newcommand{\response}{\ensuremath{\mathcal{R}}}
\newcommand{\phifid}{\ensuremath{\phi_{\text{fid}}}}
\newcommand{\ffid}{\ensuremath{f_{\text{fid}}}}
\newcommand{\tfid}{\ensuremath{t_{\text{fid}}}}
\newcommand{\tf}{\ensuremath{t_{(f)}}}
\newcommand{\tfOne}{\ensuremath{t_{1,(f)}}}
\newcommand{\tfTwo}{\ensuremath{t_{2,(f)}}}
\newcommand{\omegah}{\ensuremath{\omega_{\text{hyb}}}}
\newcommand{\fhyb}{\ensuremath{f_{\text{hyb}}}}
\newcommand{\thyb}{\ensuremath{t_{\text{hyb}}}}

%% Boldface (indexless) four-tensor notation
\newcommand{\Tensor}[1]{\bm{\mathsf{#1}}}
\newcommand{\FourVector}[1]{\bm{#1}}
\newcommand{\Vector}[1]{\FourVector{#1}}

%% Three-vector
\newcommand{\ThreeVector}[1]{\vec{#1}}

%% Polarization tensors
\newcommand{\pol}{\ensuremath{\varepsilon}}
\newcommand{\polp}{{\ensuremath{\pol_{+}}}}
\newcommand{\polc}{{\ensuremath{\pol_{\times}}}}

%% Transverse, traceless part
\newcommand{\TT}{\mathrm{TT}}


%%%%%%%%%%%%%%%%
%% Text forms %%
%%%%%%%%%%%%%%%%
\makeatletter
\newcommand{\foreign}[1]{\textit{#1}} % {\textrm{#1}}
\newcommand{\etal}{\textit{et~al}\@ifnextchar{\relax}{.\relax}{\ifx\@let@token.\else\ifx\@let@token~.\else.\@\xspace\fi\fi}}
\newcommand{\etc}{\foreign{etc}\@ifnextchar{\relax}{.\relax}{\ifx\@let@token.\else\ifx\@let@token~.\else.\@\xspace\fi\fi}}
\newcommand{\eg}{\foreign{e.g}\@ifnextchar{\relax}{.\relax}{\ifx\@let@token.\else\ifx\@let@token~.\else.\@\xspace\fi\fi}}
\newcommand{\ie}{\foreign{i.e}\@ifnextchar{\relax}{.\relax}{\ifx\@let@token.\else\ifx\@let@token~.\else.\@\xspace\fi\fi}}
\makeatother

\newcommand{\pN}{\text{PN}\xspace}
\newcommand{\PN}{\pN\xspace}
\newcommand{\Pade}{{Pad{\'{e}}}\xspace}
\newcommand{\Poincare}{Poincar{\'{e}}\xspace}
\newcommand{\Bezier}{B{\'{e}}zier\xspace}

\definecolor{NoteColor}{rgb}{0.900, 0.218, 0.000}
\newcommand{\Note}[1]{\textcolor{NoteColor}{[#1]}\index{Note}}
\definecolor{NewColor}{rgb}{0,.55,0}
\newcommand{\New}[1]{\textcolor{NewColor}{#1}\index{New}}

\newcommand{\software}[1]{\textsc{#1}}
\newcommand{\command}[1]{\texttt{\small{#1}}}
\newcommand{\code}[1]{\texttt{\small{#1}}}


%%%%%%%%%%%%%%
%% Commands %%
%%%%%%%%%%%%%%
\makeatletter
%% For caption names
\newcommand{\CapName}[1]{\textbf{#1}.}
%% The following command allows you to enter \ShowDimensions anywhere
%% in a document, and -- at processing -- see all the dimensions (I
%% could think of) in the terminal.
\newcommand{\ShowDimensions}{%
  \typeout{The font encoding is \f@encoding}        %
  \typeout{The font family is \f@family}            %
  \typeout{The font series is \f@series}            %
  \typeout{The font shape is \f@shape}              %
  \typeout{The font size is \f@size}                %
  \typeout{The baselineskip is \f@baselineskip}     %
  \typeout{The math font size is \tf@size}          %
  \typeout{The math script size is \sf@size}        %
  \typeout{The math scriptscript size is \ssf@size} %
  \typeout{The linewidth is \the\linewidth}         %
}
%% This is a simpler version of amsmath's 'sideset' command, which
%% allows prefixed sub- and superscripts.  Use as something like:
%%   $\prefixscripts{^{-2}}{Y}$
\newcommand{\prefixscripts}[2]{%
  \@mathmeasure\z@\displaystyle{#2}%
  \global\setbox\@ne\vbox to\ht\z@{}\dp\@ne\dp\z@
  \setbox\tw@\box\@ne
  \@mathmeasure4\displaystyle{\copy\tw@#1}%
  \@mathmeasure6\displaystyle{#2}%
  \dimen@-\wd6 \advance\dimen@\wd4 \advance\dimen@\wd\z@
  \hbox to\dimen@{}{\kern-\dimen@\box4\box6}%
}
%% When a symbol using \prefixscripts also has postfixed sub- and
%% superscripts, the postfixed scripts are set relatively high or low,
%% compared to the prefixed scripts.  This command improves that.  Use
%% as something like:
%%   $\scripts{^{-2}}{Y}{_{lm}}$
\newcommand{\scripts}[3]{%
  \@mathmeasure\z@\displaystyle{#2}%
  \global\setbox\@ne\vbox to\ht\z@{}\dp\@ne\dp\z@
  \setbox\tw@\box\@ne
  \@mathmeasure4\displaystyle{\copy\tw@#1}%
  \@mathmeasure6\displaystyle{#2#3}%
  \dimen@-\wd6 \advance\dimen@\wd4 \advance\dimen@\wd\z@
  \hbox to\dimen@{}{\kern-\dimen@\box4\box6}%
}
\makeatother


%%%%%%%%%%%%%%%%%%%%%%%%%%%%%%%%%%%%%%%%%%%%%%%%%%%%%%%%%%%%%%%%%%
%% My extension of the braket package for inner products        %%
%% This allows code like $\InnerProduct{s|h}$, which works much %%
%% like $\Braket{x|y}$.                                         %%
%%%%%%%%%%%%%%%%%%%%%%%%%%%%%%%%%%%%%%%%%%%%%%%%%%%%%%%%%%%%%%%%%%
\makeatletter
\let\protect\relax
{\catcode`\|=\active
  \xdef\InnerProduct{\protect\expandafter\noexpand\csname InnerProduct \endcsname}
  \expandafter\gdef\csname InnerProduct \endcsname#1{%
    \begingroup
    \ifx\SavedDoubleVert\relax
    \let\SavedDoubleVert\|\let\|\IpDoubleVert
    \fi
    \mathcode`\|32768\let|\IPVert
    \left({#1}\right)
    \endgroup
  }
}
\def\IPVert{\@ifnextchar|{\|\@gobble}% turn || into \|
     {\egroup\,\mid@vertical\,\bgroup}}
\def\IPDoubleVert{\egroup\,\mid@dblvertical\,\bgroup}
\let\SavedDoubleVert\relax
\def\midvert{\egroup\mid\bgroup}
\def\SetVert{\@ifnextchar|{\|\@gobble}% turn || into \|
    {\egroup\;\mid@vertical\;\bgroup}}
\def\SetDoubleVert{\egroup\;\mid@dblvertical\;\bgroup}
\def\mid@vertical{\mskip1mu\vrule\mskip1mu}
\def\mid@dblvertical{\mskip1mu\vrule\mskip2.5mu\vrule\mskip1mu}
\makeatother
\newcommand{\Overlap}{\Braket}
\newcommand{\Mismatch}{\ensuremath{\text{MM}}}
\newcommand{\MM}[2]{\ensuremath{\Mismatch\left(#1,#2\right)}}
\newcommand{\TargetMismatch}{\ensuremath{\Mismatch_{\text{target}}}}

%%% Local Variables:
%%% mode: latex
%%% TeX-master: "paper"
%%% End:

%auto-ignore

\newcommand{\Caltech}{\affiliation{Theoretical Astrophysics 350-17,
    California Institute of Technology, Pasadena, California 91125,
    USA}} %

\newcommand{\Cornell}{\affiliation{Center for Radiophysics and
    Space Research, Cornell University, Ithaca, New York 14853, USA}} %

\newcommand{\CITA}{\affiliation{Canadian Institute for Theoretical
    Astrophysics, University of Toronto, 60 Saint George Street,
    Toronto, Ontario M5S 3H8, Canada}} %



%% Force reprocessing of all plots
% \tikzset{external/force remake}

%%%%%%%%%%%%%%%%

\begin{document}

%%%%%%%%%%%%%%%%

\graphicspath{%
  {Plots/}%
  % More directories are added in braces, without commas between
}

\title[Short title appearing in subsequent headers] {Long title that
  appears at the beginning of the paper}

\makeatletter
\@booleantrue\frontmatterverbose@sw
\makeatother

\surnames{Boyle, Throop, Eh, \protect\etal}
\author{Michael Boyle} \Cornell
\author{Amos Throop} \Caltech
\author{Some Canuck Eh} \CITA

\date{\today}

\begin{abstract}
  Exciting, mind-blowing abstract.  Keep things short and simple, but
  describe everything interesting in the paper.  Give context for the
  work briefly, but don't use citations within the abstract.  Also,
  this is the one place where passive voice is preferred; don't say
  things like ``we do this''.  (That goes double for single-author
  papers.)  Instead, say things like ``it is shown that this paper is
  awesome and its authors totally rock.''  Go easy on abbreviations;
  unless you need to use an abbreviation multiple times in the
  abstract, wait to introduce it until the body of the paper.  Most
  importantly, be sure to advertise any new or important techniques
  and explain the conclusions of the paper.
\end{abstract}

\pacs{%
  04.30.-w, % Gravitational waves
  04.80.Nn, % Gravitational wave detectors and experiments
  04.25.D-, % Numerical relativity
  04.25.dg  % NR studies of black holes and black-hole binaries
}

% 04.25.-g, % Approximation methods; equations of motion
% 04.25.D-, % Numerical relativity
% 04.25.dc, % NR studies of crit. behavior, sing.'s, cosmic censorsh.
% 04.25.dg, % NR studies of black holes and black-hole binaries
% 04.25.dk, % NR studies of other relativistic binaries
% 04.25.Nx, % PN approximation; perturbation theory; etc.
% 04.30.-w, % Gravitational waves
% 04.30.Db, % Wave generation and sources
% 04.30.Nk, % Wave propagation and interactions
% 04.30.Tv, % Gravitational-wave astrophysics
% 04.80.Nn, % Gravitational wave detectors and experiments

\maketitle

%%%%%%%%%%%%%%%%%%%%%%%%%%%%%%%%%%%%%%%%%%%%%%%%%%%%%%%%%%%%%%%%%%%%%%
%%%%%%%%%%%%%%%%%%%%%%%%%%%%%%%%%%%%%%%%%%%%%%%%%%%%%%%%%%%%%%%%%%%%%%
\section{Introduction}
\label{sec:Introduction}

Some really cool introduction, explaining the context of this work,
why it needs to be done, how it will change the world, etc.  Grab the
reader's attention so s/he cares enough to keep reading.  Cite
appropriate sources~\cite{MTW, Wald:1984} for the context, including
any other approaches to this particular
problem~\cite{doranlasenby:2003}.

Explain how this paper attacks that problem (and why other approaches
need to be improved, if relevant).  Summarize everything that is done
in the paper, including the conclusions.  The introduction should sort
of be a mini version of the whole paper, without going into all the
details, because most readers will only get this far.

Don't be afraid of giving away the punchline here---this paper isn't a
story with dramatic twists and turns; it's meant to communicate the
ideas effectively.  Aristotle supposedly advised: ``tell them what you
are going to tell them, tell them, then tell them what you told
them.''  That should describe the structure of the paper, and even of
sections within the paper.  Don't let readers get lost wondering why
you're going through all these steps.  But try to balance that with
not repeating things too much.  For example, in the introduction,
explain at a broad level what you're going to do in
Sec.~\ref{sec:AnotherCleverOne}.  Then, at the beginning of
Sec.~\ref{sec:AnotherCleverOne}, go into more detail so that someone
can get an idea of what that section's task is and how you accomplish
the task of that section without reading through the whole thing.  At
the end of that section, briefly recap the point, to show your reader
that you actually had one.

Try to make the paper as modular as possible.  People will skip
sections, so if the motivation for Sec.~\ref{sec:AnotherCleverOne} is
contained in Sec.~\ref{sec:BrilliantSection}, don't assume the reader
saw that, understood it, or remembers it.  But again, try to minimize
repetition; if necessary, just refer to the other section with a brief
sentence or paragraph explaining the important point.  Also, begin and
end each section with a brief segue to/from the neighboring section;
don't present the paper as a series of disconnected ideas.


%%%%%%%%%%%%%%%%%%%%%%%%%%%%%%%%%%%%%%%%%%%%%%%%%%%%%%%%%%%%%%%%%%%%%%
\begin{figure}
%   \includegraphics[width=\linewidth]{SomeFigure}
  \caption{ \label{fig:SomeFigure} %
    \CapName{Caption title} Caption content. Pay a lot of attention to
    figures and captions.  Most people will skim the figures when
    trying to decide whether this paper is worth reading and
    remembering, so make every figure count.  Make the figures, axis
    labels, and captions clear and self-contained, because the reader
    will not have read the relevant text.  Also, make sure font sizes
    are comparable to the font sizes in the caption, so that things
    are readable. %
  }
\end{figure}
%%%%%%%%%%%%%%%%%%%%%%%%%%%%%%%%%%%%%%%%%%%%%%%%%%%%%%%%%%%%%%%%%%%%%%


%%%%%%%%%%%%%%%%%%%%%%%%%%%%%%%%%%%%%%%%%%%%%%%%%%%%%%%%%%%%%%%%%%%%%%
%%%%%%%%%%%%%%%%%%%%%%%%%%%%%%%%%%%%%%%%%%%%%%%%%%%%%%%%%%%%%%%%%%%%%%
\section{Brilliant section title}
\label{sec:BrilliantSection}
In this section, we use garblehobber midimuxers to show that
floobleflakkers have warlywursters.  This will be crucial to
understanding the emission of gamma rays by hockledockers in
Sec.~\ref{sec:AnotherCleverOne}.

\subsection{Subsection with \texorpdfstring{$\Sigma$}{Sigma}
  math in the title}
\label{sec:ImportantSubsection}

Use \verb|\texorpdfstring| to get rid of compiler warnings when pdf
bookmarks would complain about having math in the text of section
titles.

\subsection{More important subsection}
\label{sec:MoreImportantSubsection}
Never have just one subsection.  That would be dumb.


%%%%%%%%%%%%%%%%%%%%%%%%%%%%%%%%%%%%%%%%%%%%%%%%%%%%%%%%%%%%%%%%%%%%%%
%%%%%%%%%%%%%%%%%%%%%%%%%%%%%%%%%%%%%%%%%%%%%%%%%%%%%%%%%%%%%%%%%%%%%%
\section{Another clever title}
\label{sec:AnotherCleverOne}
In Sec.~\ref{sec:BrilliantSection}, we showed that floobleflakkers
have warlywursters.  By convolving with the spectrum of sturlywurlers,
we can understand why the spectral index of hockledockers is $-3.5$.

Lots of gibberish will be found here.

Because the PSD of sturlywurlers goes as $f^{-1.75}$, and because
gamma-ray emission is independently flabbergasted, we see that
hockledockers must have a spectral index of $-3.5$.


%%%%%%%%%%%%%%%%%%%%%%%%%%%%%%%%%%%%%%%%%%%%%%%%%%%%%%%%%%%%%%%%%%%%%%
%%%%%%%%%%%%%%%%%%%%%%%%%%%%%%%%%%%%%%%%%%%%%%%%%%%%%%%%%%%%%%%%%%%%%%
\section{Conclusions}
\label{sec:Conclusions}
Summarize what has happened in this paper, and how it will change the
world.  Look forward to future work that needs to be done on this.


%%%%%%%%%%%%%%%%%%%%%%%%%%%%%%%%%%%%%%%%%%%%%%%%%%%%%%%%%%%%%%%%%%%%%%
\begin{acknowledgments}
  It is my pleasure to thank lots of people.  This project was
  supported in part by a grant from the Sherman Fairchild Foundation;
  by NSF Grants No.\ PHY-0969111 and No.\ PHY-1005426; and by NASA
  Grant No.\ NNX09AF96G. The numerical computations presented in this
  paper were performed primarily on the \texttt{Zwicky} cluster hosted
  at Caltech by the Center for Advanced Computing Research, which was
  funded by the Sherman Fairchild Foundation and the NSF
  MRI-R\textsuperscript{2} program.
\end{acknowledgments}




%%%%%%%%%%%%%%%%%%%%%%%%%%%%%%%%%%%%%%%%%%%%%%%%%%%%%%%%%%%%%%%%%%%%%%
%%%%%%%%%%%%%%%%%%%%%%%%%%%%%%%%%%%%%%%%%%%%%%%%%%%%%%%%%%%%%%%%%%%%%%
\appendix* % Use \appendix* if there is just one appendix
% \appendix % Use \appendix if there are multiple appendices


\section{Grubby details that are really too long for the body but are
  important from a scientific standpoint}
\label{sec:GrubbyDetails}

If some poor grad student wants to implement your methods ten years
from now, or understand some long proof, the information needs to be
contained in the paper.  But putting everything in the body of the
paper would be asking people to read the boring stuff, and they'd fall
asleep.  Hence appendices.  No one bothers to read this stuff unless
they really need to, so no one falls asleep.

Also consider attaching code in ancillary files (on the arXiv) and/or
supplements (in most journals).  This is good for science and good for
your career.

%%%%%%%%%%%%%%%%%%%%%%%%%%%%%%%%%%%%%%%%%%%%%%%%%%%%%%%%%%%%%%%%%%%%%%
%%%%%%%%%%%%%%%%%%%%%%%%%%%%%%%%%%%%%%%%%%%%%%%%%%%%%%%%%%%%%%%%%%%%%%
%% References

%% Try this if the last two columns before the bib don't break nicely
% \vspace{0.1in}

\vfil

%% Try this if missing footnote on page with references:
% \clearpage

%% Bibtex sometimes uses these automatically, so we have to make sure
%% they have their original meanings when it does
\let\c\Originalcdefinition
\let\d\Originalddefinition
\let\i\Originalidefinition

%% Include any .bib files that need to be referenced, with `.bib`
%% removed, separated by commas without spaces
\bibliography{AnalyticalWaveforms,Astro,Detection,General,GeometricAlgebra,NumericalRelativity,ParameterEstimation,Scri}

%%%%%%%%%%%%%%

\end{document}

%%%%%%%%%%%%%%


%%% Local Variables:
%%% mode: latex
%%% TeX-master: t
%%% End:
