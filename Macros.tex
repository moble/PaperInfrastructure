%auto-ignore

%%%%%%%%%%%%%%%%
%% Math forms %%
%%%%%%%%%%%%%%%%
\let\Originalddefinition\d
\renewcommand{\d}{\ensuremath{\mathrm{d}}}
\let\Originaledefinition\e
\newcommand{\e}{\ensuremath{\mathrm{e}}}
\let\Originalidefinition\i
\renewcommand{\i}{\ensuremath{\mathrm{i}}}

%% Units
\let\Originalcdefinition\c
\renewcommand{\c}{\mathrm{c}}
\newcommand{\G}{\mathrm{G}}
\newcommand{\MSun}{\ensuremath{M_\odot}\xspace}
\newcommand{\Sun}{\MSun}
\newcommand{\Mtot}{\ensuremath{M_\text{tot}}}
\DeclareSIUnit{\strain}{strain}
\DeclareSIPrePower{\root}{1/2}
\DeclareSIUnit{\parsec}{pc}
\DeclareSIUnit{\yr}{yr}
\DeclareSIUnit{\year}{yr}
\DeclareSIUnit{\lightyear}{ly}
\DeclareSIUnit{\SolarMass}{\ensuremath{\MSun}}
\DeclareSIUnit{\Mass}{\ensuremath{M}}
% \newcommand{\strain}{\text{strain}}
% \newcommand{\parsec}{\text{pc}}
% \newcommand{\SolarMass}{\MSun}

%% abs, norm, ceil, floor, avg, etc.
\newcommand{\abs} [1]{\left\lvert{#1}\right\rvert}
\newcommand{\norm}[1]{\left\lVert{#1}\right\rVert}
\newcommand{\ceil}[1]{\left\lceil{#1}\right\rceil}
\newcommand{\floor}[1]{\left\lfloor{#1}\right\rfloor}
\newcommand{\avg}[1]{\left\langle{#1}\right\rangle}
\newcommand{\co}[1]{\ensuremath{\bar{#1}}}
\DeclareMathOperator*{\RS}{RiemannSymmetrize}
\DeclareMathOperator{\sgn}{sgn}
\DeclareMathOperator*{\argmin}{arg\,min}
\DeclareMathOperator*{\argmax}{arg\,max}
% \DeclareMathOperator{\arcsin}{arcsin}
% \DeclareMathOperator{\arccos}{arccos}
% \DeclareMathOperator{\arctan}{arctan}
\DeclareMathOperator{\arccsc}{arccsc}
\DeclareMathOperator{\arcsec}{arcsec}
\DeclareMathOperator{\arccot}{arccot}
\DeclareMathOperator{\asinh}{asinh}
\DeclareMathOperator{\acosh}{acosh}
\DeclareMathOperator{\atanh}{atanh}
\DeclareMathOperator{\acsch}{acsch}
\DeclareMathOperator{\asech}{asech}
\DeclareMathOperator{\acoth}{acoth}

%% Relational operators
\newcommand{\defined}{\coloneqq}
\newcommand{\identically}{\equiv}
\newcommand{\roughly}{\mathord{\sim}} % Different from \sim in spacing
%\newcommand{\corresponds}{\triangleq}
\newcommand{\asymptoticallyequal}{\simeq}
\newcommand{\homeomorphic}{\cong}
\newcommand{\isomorphic}{\approx}



%% Asymptotically flat limits
\newcommand{\scri}{\ensuremath{\mathscr{I}}}
\newcommand{\scriplus}{\ensuremath{\mathscr{I}^{+}}}
\newcommand{\scriminus}{\ensuremath{\mathscr{I}^{-}}}
\newcommand{\Mirr}{\ensuremath{M_{\text{irr}}}}
\newcommand{\Eadm}{\ensuremath{M_{\text{ADM}}}}
\newcommand{\Jadm}{\ensuremath{J_{\text{ADM}}}}
\newcommand{\ADMMass}{\Eadm}
\newcommand{\IrrMass}{\Mirr}
\newcommand{\tr}{\ensuremath{t_\text{ret}}}
\newcommand{\rt}{\ensuremath{r_{\ast}}}

%% Wigner matrices, rotations, etc.
\newcommand{\trivialgroup}{\langle 0 \rangle}
\newcommand{\D}{\ensuremath{\mathfrak{D}} }
\newcommand{\R}{\ensuremath{\mathcal{R}} }
% \newcommand{\sYlm}[1]{\ensuremath{\prescript{}{s}{Y}_{#1}}}
% \newcommand{\mTwoYlm}[1]{\ensuremath{\prescript{}{-2}{Y}_{#1}}}
\newcommand{\sYlm}[1]{\ensuremath{\scripts{_{s}}{Y}{_{#1}}}}
\newcommand{\sYbarlm}[1]{\ensuremath{\scripts{_{s}}{\bar{Y}}{_{#1}}}}
\newcommand{\mTwoYlm}[1]{\ensuremath{\scripts{_{-2}}{Y}{_{#1}}}}
\newcommand{\mTwoYbarlm}[1]{\ensuremath{\scripts{_{-2}}{\bar{Y}}{_{#1}}}}
%\newcommand{\Rotated}[1]{\ensuremath{\accentset{\frown}{#1}}}
%\newcommand{\Rotated}[1]{\ensuremath{\breve{#1}}}
%\newcommand{\Rotated}[1]{\ensuremath{#1'}}
\DeclareSymbolFont{tipa}{T3}{tipa}{m}{n}
\DeclareMathAccent{\ibreve}{\mathalpha}{tipa}{'020}
\newcommand{\Rotated}[1]{\ensuremath{\ibreve{#1}}}
\newcommand{\Rotation}[1]{\mathbf{\MakeUppercase{#1}}}
\newcommand{\LieElement}[1]{\mathbf{\MakeUppercase{#1}}}
\newcommand{\LieComponent}[2]{\mathrm{\MakeUppercase{#1}}_{#2}}
\newcommand{\LieGenerator}[1]{\mathbf{\MakeLowercase{#1}}}
\newcommand{\LieGeneratorMag}[1]{\mathrm{\MakeLowercase{#1}}}
\newcommand{\GeneratorVec}[1]{\ensuremath{\vec{\MakeLowercase{#1}}}}
\DeclareMathOperator{\adjoint}{ad}
\DeclareMathOperator{\Adjoint}{Ad}
\newcommand{\ad}[1]{\adjoint_{\LieGenerator{#1}}}
\newcommand{\Ad}[1]{\Adjoint_{\LieGenerator{#1}}}
\newcommandx{\structure}[3][2={}, 3={}]{f^{#1 #2}_{#3}}

\newcommand{\SO}[1]{\ensuremath{\mathrm{SO}(#1)}}
\newcommand{\so}[1]{\ensuremath{\mathfrak{so}(#1)}}
\newcommand{\SOp}[1]{\ensuremath{\mathrm{SO}^{+}(#1)}}
\newcommand{\sop}[1]{\ensuremath{\mathfrak{so}^{+}(#1)}}
\newcommand{\SU}[1]{\ensuremath{\mathrm{SU}(#1)}}
\newcommand{\su}[1]{\ensuremath{\mathfrak{su}(#1)}}
\newcommand{\Spin}[1]{\ensuremath{\mathrm{Spin}(#1)}}
\newcommand{\spin}[1]{\ensuremath{\mathfrak{so}(#1)}}
\newcommand{\HopfMap}{\mathfrak{h}}
\newcommand{\TangentBundle}[1]{\mathrm{T}\left(#1\right)}
\newcommand{\UnitTangentBundle}[1]{\mathrm{UT}\left(#1\right)}
\newcommand{\TangentBundleAt}[2]{\mathrm{T}_{#1}\left(#2\right)}
\newcommand{\UnitTangentBundleAt}[2]{\mathrm{UT}_{#1}\left(#2\right)}

%% Metric perturbation
\newcommand{\Background}[1]{\underaccent{\frown}{#1}}
\newcommand{\h}{\ensuremath{h} }
\newcommand{\hp}{\ensuremath{h_{+}} }
\newcommand{\hc}{\ensuremath{h_{\times}} }
\newcommand{\htilde}{\ensuremath{{\tilde{h}}}}
\newcommand{\Schw}{\ensuremath{\Background{R}} }
\newcommand{\GW}{\tilde{R}}
\newcommand{\Schwg}{\ensuremath{\Background{g}} }
\newcommand{\GWg}{\tilde{g}}
\newcommand{\SchwScalar}{\ensuremath{{\Background{\boxdot}}} }

%% Frequencies, discretization, DA stuff...
\newcommand{\fsamp}{\ensuremath{f_{\mathrm{s}}}}
\newcommand{\fNy}{\ensuremath{f_{\mathrm{Ny}}}}
\newcommand{\discretize}{\rightsquigarrow}
%%\newcommand{\MM}{\text{MM}} %% See below
\newcommand{\parameters}{\vec{\lambda}}
\newcommand{\response}{\ensuremath{\mathcal{R}}}
\newcommand{\phifid}{\ensuremath{\phi_{\text{fid}}}}
\newcommand{\ffid}{\ensuremath{f_{\text{fid}}}}
\newcommand{\tfid}{\ensuremath{t_{\text{fid}}}}
\newcommand{\tf}{\ensuremath{t_{(f)}}}
\newcommand{\tfOne}{\ensuremath{t_{1,(f)}}}
\newcommand{\tfTwo}{\ensuremath{t_{2,(f)}}}
\newcommand{\omegah}{\ensuremath{\omega_{\text{hyb}}}}
\newcommand{\fhyb}{\ensuremath{f_{\text{hyb}}}}
\newcommand{\thyb}{\ensuremath{t_{\text{hyb}}}}

%% Boldface (indexless) four-tensor notation
\newcommand{\Tensor}[1]{\bm{\mathsf{#1}}}
\newcommand{\FourVector}[1]{\bm{#1}}
\newcommand{\Vector}[1]{\FourVector{#1}}

%% Three-vector
\newcommand{\ThreeVector}[1]{\Vector{#1}}

\newcommand{\fourvec}[1]{\FourVector{#1}}
% \renewcommand{\threevec}[1]{\ensuremath{\vec{#1}}}
% \newcommand{\threevec}{\fourvec}
% \newcommand{\unitvec}[1]{\hat{\fourvec{#1}}}
\newcommand{\threevec}{\Vector}
\newcommand{\rHat}{\unitvec{r}}
\newcommand{\nHat}{\unitvec{n}}
\newcommand{\xHat}{\unitvec{x}}
\newcommand{\yHat}{\unitvec{y}}
\newcommand{\zHat}{\unitvec{z}}

%% Polarization tensors
\newcommand{\pol}{\ensuremath{\varepsilon}}
\newcommand{\polp}{{\ensuremath{\pol_{+}}}}
\newcommand{\polc}{{\ensuremath{\pol_{\times}}}}

%% Transverse, traceless part
\newcommand{\TT}{\mathrm{TT}}


%%%%%%%%%%%%%%%%
%% Text forms %%
%%%%%%%%%%%%%%%%
\makeatletter
\newcommand{\foreign}[1]{\textit{#1}} % {\textrm{#1}}
\newcommand{\etal}{\textit{et~al}\@ifnextchar{\relax}{.\relax}{\ifx\@let@token.\else\ifx\@let@token~.\else.\@\xspace\fi\fi}}
\newcommand{\etc}{\foreign{etc}\@ifnextchar{\relax}{.\relax}{\ifx\@let@token.\else\ifx\@let@token~.\else.\@\xspace\fi\fi}}
\newcommand{\eg}{\foreign{e.g}\@ifnextchar{\relax}{.\relax}{\ifx\@let@token.\else\ifx\@let@token~.\else.\@\xspace\fi\fi}}
\newcommand{\ie}{\foreign{i.e}\@ifnextchar{\relax}{.\relax}{\ifx\@let@token.\else\ifx\@let@token~.\else.\@\xspace\fi\fi}}
\newcommand{\perse}{\foreign{per~se}\xspace}
\makeatother

\newcommand{\pN}{\text{PN}\xspace}
\newcommand{\PN}{\pN\xspace}
\newcommand{\Pade}{{Pad{\'{e}}}\xspace}
\newcommand{\Poincare}{Poincar{\'{e}}\xspace}
\newcommand{\Bezier}{B{\'{e}}zier\xspace}
\newcommand{\Mobius}{M{\"{o}}bius\xspace}

\definecolor{NoteColor}{rgb}{0.900, 0.218, 0.000}
\newcommand{\Note}[1]{\textcolor{NoteColor}{[#1]}\index{Note}}
\definecolor{NewColor}{rgb}{0,.55,0}
\newcommand{\New}[1]{\textcolor{NewColor}{#1}\index{New}}

\newcommand{\software}[1]{\textsc{#1}}
\newcommand{\command}[1]{\texttt{\small{#1}}}
\newcommand{\code}[1]{\texttt{\small{#1}}}


%%%%%%%%%%%%%%
%% Commands %%
%%%%%%%%%%%%%%
\makeatletter
%% For caption names
\newcommand{\CapName}[1]{\textbf{#1}.}
%% The following command allows you to enter \ShowDimensions anywhere
%% in a document, and -- at processing -- see all the dimensions (I
%% could think of) in the terminal.
\newcommand{\ShowDimensions}{%
  \typeout{The font encoding is \f@encoding}        %
  \typeout{The font family is \f@family}            %
  \typeout{The font series is \f@series}            %
  \typeout{The font shape is \f@shape}              %
  \typeout{The font size is \f@size}                %
  \typeout{The baselineskip is \f@baselineskip}     %
  \typeout{The math font size is \tf@size}          %
  \typeout{The math script size is \sf@size}        %
  \typeout{The math scriptscript size is \ssf@size} %
  \typeout{The linewidth is \the\linewidth}         %
  \typeout{The textwidth is \the\textwidth}         %
}
%% This is a simpler version of amsmath's 'sideset' command, which
%% allows prefixed sub- and superscripts.  Use as something like:
%%   $\prefixscripts{^{-2}}{Y}$
\newcommand{\prefixscripts}[2]{%
  \@mathmeasure\z@\displaystyle{#2}%
  \global\setbox\@ne\vbox to\ht\z@{}\dp\@ne\dp\z@
  \setbox\tw@\box\@ne
  \@mathmeasure4\displaystyle{\copy\tw@#1}%
  \@mathmeasure6\displaystyle{#2}%
  \dimen@-\wd6 \advance\dimen@\wd4 \advance\dimen@\wd\z@
  \hbox to\dimen@{}{\kern-\dimen@\box4\box6}%
}
%% When a symbol using \prefixscripts also has postfixed sub- and
%% superscripts, the postfixed scripts are set relatively high or low,
%% compared to the prefixed scripts.  This command improves that.  Use
%% as something like:
%%   $\scripts{^{-2}}{Y}{_{lm}}$
\newcommand{\scripts}[3]{%
  \@mathmeasure\z@\displaystyle{#2}%
  \global\setbox\@ne\vbox to\ht\z@{}\dp\@ne\dp\z@
  \setbox\tw@\box\@ne
  \@mathmeasure4\displaystyle{\copy\tw@#1}%
  \@mathmeasure6\displaystyle{#2#3}%
  \dimen@-\wd6 \advance\dimen@\wd4 \advance\dimen@\wd\z@
  \hbox to\dimen@{}{\kern-\dimen@\box4\box6}%
}
\makeatother


%%%%%%%%%%%%%%%%%%%%%%%%%%%%%%%%%%%%%%%%%%%%%%%%%%%%%%%%%%%%%%%%%%
%% My extension of the braket package for inner products        %%
%% This allows code like $\InnerProduct{s|h}$, which works much %%
%% like $\Braket{x|y}$.                                         %%
%%%%%%%%%%%%%%%%%%%%%%%%%%%%%%%%%%%%%%%%%%%%%%%%%%%%%%%%%%%%%%%%%%
\makeatletter
\let\protect\relax
{\catcode`\|=\active
  \xdef\InnerProduct{\protect\expandafter\noexpand\csname InnerProduct \endcsname}
  \expandafter\gdef\csname InnerProduct \endcsname#1{%
    \begingroup
    \ifx\SavedDoubleVert\relax
    \let\SavedDoubleVert\|\let\|\IpDoubleVert
    \fi
    \mathcode`\|32768\let|\IPVert
    \left({#1}\right)
    \endgroup
  }
}
\def\IPVert{\@ifnextchar|{\|\@gobble}% turn || into \|
     {\egroup\,\mid@vertical\,\bgroup}}
\def\IPDoubleVert{\egroup\,\mid@dblvertical\,\bgroup}
\let\SavedDoubleVert\relax
\def\midvert{\egroup\mid\bgroup}
\def\SetVert{\@ifnextchar|{\|\@gobble}% turn || into \|
    {\egroup\;\mid@vertical\;\bgroup}}
\def\SetDoubleVert{\egroup\;\mid@dblvertical\;\bgroup}
\def\mid@vertical{\mskip1mu\vrule\mskip1mu}
\def\mid@dblvertical{\mskip1mu\vrule\mskip2.5mu\vrule\mskip1mu}
\makeatother
\newcommand{\Overlap}{\Braket}
\newcommand{\Mismatch}{\ensuremath{\text{MM}}}
\newcommand{\MM}[2]{\ensuremath{\Mismatch\left(#1,#2\right)}}
\newcommand{\TargetMismatch}{\ensuremath{\Mismatch_{\text{target}}}}

%%% Local Variables:
%%% mode: latex
%%% TeX-master: "paper"
%%% End:
