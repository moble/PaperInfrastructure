%auto-ignore

\usepackage{xparse} % For \NewDocumentCommand, etc.

\makeatletter

\newcommand{\vectorspace}[1]{\mathcal{#1}}
\newcommand{\field}[1]{\mathbb{#1}}

\newcommand{\framevec}[1]{\bm{\mathsf{#1}}}

\newcommand{\grade}[2]{\ensuremath{\left\langle #1 \right\rangle_{#2}}}
\newcommand{\scalar}[1]{\ensuremath{\left\langle #1 \right\rangle}}
\newcommand{\grading}[1]{\ensuremath{\langle #1 \rangle}}
\newcommand{\graded}[2]{\ensuremath{#1_{\grading{#2}}}}

\newcommand{\vectorderiv}{\ensuremath{\nabla}}


%% Various involutions
\newcommand{\conjug@@te}[3]{\raisebox{#1}{$#2\text{\tiny{#3}}$}}
\newcommand{\conjug@te}[3]{\overline{#1}^{\mathpalette{\conjug@@te{#3}}{#2}}}
\newcommand{\reverse}[1]{\conjug@te{\multivec{#1}}{rev}{.4pt}}
\newcommand{\involution}[1]{\conjug@te{\multivec{#1}}{invo}{.4pt}}
\newcommand{\Cc}[1]{\conjug@te{\multivec{#1}}{Cc}{.1pt}}
\newcommand{\Herm}[1]{\conjug@te{\multivec{#1}}{Herm}{.1pt}}
\newcommand{\spinconjugate}[1]{\conjug@te{\multivec{#1}}{sc}{.4pt}}


%% Define \multivector command with optional grading
\newcommand{\@multivector}[1]{\ensuremath{#1}}
\ExplSyntaxOn
\NewDocumentCommand{\multivec}{ mo }
    {
                \IfNoValueTF {#2}  { \@multivector{#1} }
                    { \graded{\@multivector{#1}}{#2} }
    }
\NewDocumentCommand{\multivecrev}{ mo }
    {
                \IfNoValueTF {#2}  { \reverse{\@multivector{#1}} }
                    { \graded{\reverse{\@multivector{#1}}}{#2} }
    }
\ExplSyntaxOff

\newcommand{\pseudoscalar}{\ensuremath{\multivec{I}}}

\newcommand{\cross}{\ensuremath{\times_{\text{c}}}}

\newcommand{\basisvec}[1]{\ensuremath{\multivec{#1}}}
% \newcommand{\threevec}[1]{\ensuremath{\multivec{#1}}}
\newcommand{\unitvec}[1]{\ensuremath{\hat{#1}}}
\newcommand{\axis}[1]{\ensuremath{\hat{#1}}}

\newcommand{\spinor}[1]{\ensuremath{\multivec{#1}}}
\newcommand{\rotor}[1]{\ensuremath{\multivec{#1}}}


% A linear function is just a general function (can take any type and
% number of arguments, and have any return value) that is linear in
% all arguments.
\newcommand{\linfunc}[1]{\ensuremath{\mathsf{#1}}}

% A linear transformation is a linear function that returns exactly
% the same grade as its input
\newcommand{\lintransf}[1]{\ensuremath{\mathsfsl{#1}}}
\DeclareMathAlphabet{\mathsfsl}{\encodingdefault}{\sfdefault}{m}{sl}

% Operators on linear functions and transformations
\newcommand{\differential}[1]{\ensuremath{\underline{#1}}}
% \newcommand{\adjoint}[1]{\ensuremath{\overline{#1}}}


% Write conditions restricting the domain of validity of the equation,
% which get typeset subtly next to the equation number.
\definecolor{lightgrey}{gray}{0.75}
\def\maketag@@@#1{\@Restriction\hbox{\m@th\normalfont#1}}
\def\Restriction#1{\def\@Restriction{\llap{\textcolor{lightgrey}{$#1$}~}}}
\Restriction{}



\makeatother


%%% Local Variables:
%%% mode: latex
%%% TeX-master: "paper"
%%% End:
